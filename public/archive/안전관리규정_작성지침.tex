% Options for packages loaded elsewhere
\PassOptionsToPackage{unicode}{hyperref}
\PassOptionsToPackage{hyphens}{url}
\PassOptionsToPackage{dvipsnames,svgnames,x11names}{xcolor}
%
\documentclass[
  letterpaper,
  DIV=11,
  numbers=noendperiod]{scrartcl}

\usepackage{amsmath,amssymb}
\usepackage{iftex}
\ifPDFTeX
  \usepackage[T1]{fontenc}
  \usepackage[utf8]{inputenc}
  \usepackage{textcomp} % provide euro and other symbols
\else % if luatex or xetex
  \usepackage{unicode-math}
  \defaultfontfeatures{Scale=MatchLowercase}
  \defaultfontfeatures[\rmfamily]{Ligatures=TeX,Scale=1}
\fi
\usepackage{lmodern}
\ifPDFTeX\else  
    % xetex/luatex font selection
    \setmainfont[]{Malgun Gothic}
\fi
% Use upquote if available, for straight quotes in verbatim environments
\IfFileExists{upquote.sty}{\usepackage{upquote}}{}
\IfFileExists{microtype.sty}{% use microtype if available
  \usepackage[]{microtype}
  \UseMicrotypeSet[protrusion]{basicmath} % disable protrusion for tt fonts
}{}
\makeatletter
\@ifundefined{KOMAClassName}{% if non-KOMA class
  \IfFileExists{parskip.sty}{%
    \usepackage{parskip}
  }{% else
    \setlength{\parindent}{0pt}
    \setlength{\parskip}{6pt plus 2pt minus 1pt}}
}{% if KOMA class
  \KOMAoptions{parskip=half}}
\makeatother
\usepackage{xcolor}
\usepackage[top=20mm,bottom=20mm,left=20mm,right=20mm]{geometry}
\setlength{\emergencystretch}{3em} % prevent overfull lines
\setcounter{secnumdepth}{-\maxdimen} % remove section numbering
% Make \paragraph and \subparagraph free-standing
\makeatletter
\ifx\paragraph\undefined\else
  \let\oldparagraph\paragraph
  \renewcommand{\paragraph}{
    \@ifstar
      \xxxParagraphStar
      \xxxParagraphNoStar
  }
  \newcommand{\xxxParagraphStar}[1]{\oldparagraph*{#1}\mbox{}}
  \newcommand{\xxxParagraphNoStar}[1]{\oldparagraph{#1}\mbox{}}
\fi
\ifx\subparagraph\undefined\else
  \let\oldsubparagraph\subparagraph
  \renewcommand{\subparagraph}{
    \@ifstar
      \xxxSubParagraphStar
      \xxxSubParagraphNoStar
  }
  \newcommand{\xxxSubParagraphStar}[1]{\oldsubparagraph*{#1}\mbox{}}
  \newcommand{\xxxSubParagraphNoStar}[1]{\oldsubparagraph{#1}\mbox{}}
\fi
\makeatother


\providecommand{\tightlist}{%
  \setlength{\itemsep}{0pt}\setlength{\parskip}{0pt}}\usepackage{longtable,booktabs,array}
\usepackage{calc} % for calculating minipage widths
% Correct order of tables after \paragraph or \subparagraph
\usepackage{etoolbox}
\makeatletter
\patchcmd\longtable{\par}{\if@noskipsec\mbox{}\fi\par}{}{}
\makeatother
% Allow footnotes in longtable head/foot
\IfFileExists{footnotehyper.sty}{\usepackage{footnotehyper}}{\usepackage{footnote}}
\makesavenoteenv{longtable}
\usepackage{graphicx}
\makeatletter
\newsavebox\pandoc@box
\newcommand*\pandocbounded[1]{% scales image to fit in text height/width
  \sbox\pandoc@box{#1}%
  \Gscale@div\@tempa{\textheight}{\dimexpr\ht\pandoc@box+\dp\pandoc@box\relax}%
  \Gscale@div\@tempb{\linewidth}{\wd\pandoc@box}%
  \ifdim\@tempb\p@<\@tempa\p@\let\@tempa\@tempb\fi% select the smaller of both
  \ifdim\@tempa\p@<\p@\scalebox{\@tempa}{\usebox\pandoc@box}%
  \else\usebox{\pandoc@box}%
  \fi%
}
% Set default figure placement to htbp
\def\fps@figure{htbp}
\makeatother

\KOMAoption{captions}{tableheading}
\makeatletter
\@ifpackageloaded{caption}{}{\usepackage{caption}}
\AtBeginDocument{%
\ifdefined\contentsname
  \renewcommand*\contentsname{Table of contents}
\else
  \newcommand\contentsname{Table of contents}
\fi
\ifdefined\listfigurename
  \renewcommand*\listfigurename{List of Figures}
\else
  \newcommand\listfigurename{List of Figures}
\fi
\ifdefined\listtablename
  \renewcommand*\listtablename{List of Tables}
\else
  \newcommand\listtablename{List of Tables}
\fi
\ifdefined\figurename
  \renewcommand*\figurename{Figure}
\else
  \newcommand\figurename{Figure}
\fi
\ifdefined\tablename
  \renewcommand*\tablename{Table}
\else
  \newcommand\tablename{Table}
\fi
}
\@ifpackageloaded{float}{}{\usepackage{float}}
\floatstyle{ruled}
\@ifundefined{c@chapter}{\newfloat{codelisting}{h}{lop}}{\newfloat{codelisting}{h}{lop}[chapter]}
\floatname{codelisting}{Listing}
\newcommand*\listoflistings{\listof{codelisting}{List of Listings}}
\makeatother
\makeatletter
\makeatother
\makeatletter
\@ifpackageloaded{caption}{}{\usepackage{caption}}
\@ifpackageloaded{subcaption}{}{\usepackage{subcaption}}
\makeatother

\usepackage{bookmark}

\IfFileExists{xurl.sty}{\usepackage{xurl}}{} % add URL line breaks if available
\urlstyle{same} % disable monospaced font for URLs
\hypersetup{
  pdftitle={안전관리규정 작성지침},
  colorlinks=true,
  linkcolor={blue},
  filecolor={Maroon},
  citecolor={Blue},
  urlcolor={Blue},
  pdfcreator={LaTeX via pandoc}}


\title{안전관리규정 작성지침}
\author{}
\date{}

\begin{document}
\maketitle


\textbf{원자력안전법 시행규칙 제58조제5항 관련 주요 작성 항목}

\begin{enumerate}
\def\labelenumi{\arabic{enumi}.}
\item
  \textbf{방사성동위원소 등 또는 방사성동위원소에 의하여 오염된 물질을
  취급하는 조직 및 그 기능에 관한 사항} : 방사선안전관리를 위한 조직도와
  함께 조직도에 명시된 관련자의 직무를 기술한다.
\item
  \textbf{방사성동위원소 등의 구매·사용 및 판매에 관한 사항} :
  방사성동위원소 등의 구매·사용 및 판매(생산하여 판매하는 경우에는
  생산단계를 포함한다)하기 위한 절차 및 관리방법을 기술한다.
\item
  \textbf{방사성동위원소 또는 방사성동위원소에 의하여 오염된 물질의
  분배·보관·운반·처리·배출·저장·자체처분 및 인도에 관한 사항} :
  방사성동위원소 또는 방사성동위원소에 의하여 오염된 물질의
  분배·보관·운반·처리·배출·저장·자체처분 및 인도에 관한 절차 및 기준을
  기술한다.
\item
  \textbf{방사선량률·피폭방사선량 및 방사성물질 또는 그에 의하여 오염된
  물질에 따른 오염상황의 측정 및 그 측정결과의 기록과 보존에 관한 사항}
  : 측정의 대상 또는 장소 및 주기와 그 결과의 기록사항 및 보존기간에
  대하여 기술하고 관련 양식을 첨부한다.
\item
  \textbf{방사선안전관리장비의 보관·관리 및 교정에 관한 사항} :
  방사선안전관리장비의 보관 및 관리방법과 교정주기에 대하여 기술한다.
\item
  \textbf{방사선작업종사자 및 수시출입자 분류 대상을 각각 세부적으로
  기술한다.}
\end{enumerate}

6의2. \textbf{방사선작업종사자 및 수시출입자의 피폭방사선량의 평가 및
개인선량계의 관리에 관한 사항} : 방사선작업종사자 및 수시출입자
피폭방사선량을 평가하기 위한 개인선량계의 패용방법 및 절차와
개인선량계의 관리 및 판독 또는 교정주기에 대하여 기술한다.

\begin{enumerate}
\def\labelenumi{\arabic{enumi}.}
\setcounter{enumi}{6}
\item
  \textbf{방사선작업종사자 또는 수시출입자의 방사선장해발생을 방지하기
  위하여 필요한 교육훈련에 관한 사항} : 교육훈련의 대상, 주기 및
  교과목의 내용과 교육훈련을 수행하는 절차 및 결과의 관리에 대하여
  기술한다.
\item
  \textbf{방사선장해발생 여부를 발견하기 위하여 필요한 조치에 관한 사항}
  : 방사선장해발생 여부를 발견하기 위한 건강진단 등 조치사항에 대하여
  기술한다.
\item
  \textbf{방사선장해를 받은 자 또는 그 우려가 있는 자에 대하여 취하여야
  할 보건상 조치에 관한 사항} : 방사선 장해발생자 또는 우려자에 대한
  구체적인 조치사항에 대하여 기술한다.
\item
  \textbf{「원자력안전법」제58조에 따른 기록과 이의 비치에 관한 사항} :
  원자력안전법령 및 관련 고시에서 규정하고 있는 각종 기록사항 및
  보존기간을 기술하고 관련양식을 첨부한다.
\item
  \textbf{위험시의 조치에 관한 사항} : 생산 또는 사용하고자 하는
  방사성동위원소 등과 관련하여 비상사태를 대비한 응급조치, 보고 및
  조치사항에 대하여 기술한다.
\item
  \textbf{방사성동위원소 등의 분실·도난 등 사고시의 조치 및 사고예방에
  관한 사항} : 방사성동위원소 등의 분실 또는 도난 등의 사고방지를 위한
  구체적인 수행계획과 사고발생시의 조치사항에 대하여 기술한다.
\item
  \textbf{방사선안전관리자의 권한·책임 및 직무수행에 관한 사항} :
  방사선안전관리에 관한 방사선안전관리자의 권한, 책임 및 직무수행에
  관하여 구체적으로 기술한다. : 안전관리자 대리자 지정에 대한 내용 및
  지정서양식을 포함한다(2019.2.8.\textasciitilde)
\item
  \textbf{기타 방사선장해의 방어에 필요한 사항} : 사용하고자 하는
  방사성동위원소 등의 특성에 따라 별도로 안전관리가 요구되는 사항이 있는
  경우에는 이에 관하여 기술한다.
\end{enumerate}

\begin{center}\rule{0.5\linewidth}{0.5pt}\end{center}

☞ 위 항목에 대한 세부 작성지침은
\href{https://www.law.go.kr/행정규칙/방사성동위원소등의안전관리규정작성지침}{『원자력안전위원회
고시 안전관리규정 작성지침』} 참조




\end{document}
